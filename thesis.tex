\documentclass[12pt,a4paper,twoside,draft]{book}

% use Libertine font
\usepackage{libertine}
\usepackage[T1]{fontenc}

\usepackage[utf8]{inputenc}
\usepackage[inline]{enumitem}
\usepackage{parskip} % disable indentation for new paragraphs, increased margin-bottom instead
\usepackage[ngerman,american]{babel}
\usepackage{csquotes}

\usepackage{kit_style/kitthesiscover}

\usepackage[style=alphabetic]{biblatex}
\addbibresource{bib.bib}
\setcounter{biburlnumpenalty}{100}
\setcounter{biburllcpenalty}{7000}
\setcounter{biburlucpenalty}{8000}

\usepackage{todonotes}
\usepackage{blindtext}

\usepackage{xparse}

\usepackage{xspace}

\usepackage{listings}

% for core allocation pseudo-code mostly...
\usepackage{amsmath}
\usepackage{algorithmicx}
\usepackage{varwidth}
\usepackage{calc} % for \widthof
%\usepackage{algorithm}
\usepackage{algpseudocode}
\usepackage{mathtools}
\DeclarePairedDelimiter\ceil{\lceil}{\rceil}
\DeclarePairedDelimiter\floor{\lfloor}{\rfloor}
\usepackage{amsfonts}
\usepackage{setspace}

% pandas to_latex tables look good this way
\usepackage{booktabs}
\usepackage{multirow}

\usepackage{subcaption}

\usepackage{placeins}

\usepackage{hyperref}

\widowpenalty100000
\clubpenalty100000
\raggedbottom

\begin{document}
\frontmatter
\unitlength1cm
\selectlanguage{american}

\title{Low-Latency Synchronous I/O For OpenZFS Using Persistent Memory}
\author{Christian Schwarz}
\thesistype{ma}
\primaryreviewer{Prof.\ Dr.\ Frank Bellosa}
\advisor{M.\ Sc.\ Lukas Werling}{}
\thesisbegindate{TODO}
\thesisenddate{TODO}
\maketitle

\thispagestyle{empty}
\vspace*{30\baselineskip}
%\hbox to \textwidth{\hrulefill}
\par
\iflanguage{ngerman}{
\noindent Ich versichere wahrheitsgem"a"s, die Arbeit selbstst"andig verfasst, alle benutzten Hilfsmittel vollst"andig und genau angegeben und alles kenntlich gemacht zu haben, was aus Arbeiten anderer unver"andert oder mit Ab"anderungen entnommen wurde sowie die Satzung des KIT zur Sicherung guter wissenschaftlicher Praxis in der jeweils g"ultigen Fassung beachtet zu haben.\\

\noindent Karlsruhe, den \declaration@date
}{
\noindent I hereby declare that the work presented in this thesis is entirely my own and that I did not use any source or auxiliary means other than these referenced. This thesis was carried out in accordance with the Rules for Safeguarding Good Scientific Practice at Karlsruhe Institute of Technology (KIT).\\

\noindent Karlsruhe, \declaration@date
}

%%%%%%%%%%%%%%%%%%%%%%%%%%%%%%%%%%%%%%%%%%%%%%%%%%%%%%%%%%%%%%%%%%%%%%%%
%% Hinweis:
%%
%% Diese Erklärung wird von der Prüfungsordnung für Diplomarbeiten 
%% verlangt und ist zu unterschreiben. Für Studienarbeiten ist diese
%% Erklärung nicht zwingend notwendig, schadet aber auch nicht.
%%%%%%%%%%%%%%%%%%%%%%%%%%%%%%%%%%%%%%%%%%%%%%%%%%%%%%%%%%%%%%%%%%%%%%%%
\clearpage








\chapter{Abstract}

\mainmatter
\cleardoublepage
\phantomsection
\addcontentsline{toc}{chapter}{Contents}
\tableofcontents

\chapter{Introduction}

\chapter{Background}
\section{Synchronous I/O}
\section{Persistent Memory}
\section{OpenZFS}

\chapter{Why ZIL-LWB Is Slow On PMEM}

\chapter{ZIL-PMEM: Design \& Implementation}

\section{Goals \& Non-Goals}
\section{High-Level Overview}
We introduce the concept of ZIL kinds to ZFS.
The ZIL kind is a pool-scoped variable that determines the strategy for persisting ZIL entries.
A zpool's ZIL kind is determined by the following rule:
if the pool has exactly one SLOG and that SLOG is PMEM, the ZIL kind is "ZIL-PMEM". Oterwise, it is "ZIL-LWB".
ZIL-LWB is the current ZIL's persistence implementation. It uses the SPA's metaslab allocator to allocate log-write blocks ("LWB"s) from the storage pool with a bias towards SLOG devices.
ZIL-PMEM disables metaslab allocation for its PMEM SLOG deviceand uses the resulting free PMEM space directy.

The PMEM space is partitioned into fixed-size segments.
Each segment has a corresponding in-DRAM data structure called "chunk" that tracks the segment's kernel-virtual base address and length.
Chunks are organized in a pool-wide in-DRAM data structure called PRB.
The PRB is the central code module of ZIL-PMEM.
PRB implements a high-performance crash-consistent, scalable, append-only, data-corruption-checking and garbage-collected write-ahead log that uses the PMEM chunks' segments for persistence.
After a system crash, a new instance of PRB is instantiated with the same chunks/segments that were used by the PRB instance that wrote the log before the crash.
A chunk that contains non-obsolete ("claimed") log entries is not re-used for logging until all of its entries have been obsoleted by replay.

PRB is a pool-wide resource but the ZIL is a per-head-dataset log.
Each ZIL is written replayed independently.
PRB thus maintains a per-log in-DRAM data structure called HDL in which it stores per-log state such as the log's globally unique ID, claims on chunks and replay progress.
HDLs are allocated for each of the pool's datasets on pool import.
When a dataset is created or destroyed, its HDL is allocated or freed as well.
HDL state persisted to the head dataset's ZIL header during txg sync, allowing for crash-safe replay.

PRB log entries consist of an opaque variable-sized body and a plain fixed-size header.
A log entry is always written to exactly one of the PRBs chunks' PMEM segment.
The PMEM segment is organized as a contiguous append-only sequence of entries.
Starting at offset zero, the sequence can be traversed using the length information stored in the entry header.
It is terminated by an invalid header.
If a chunk's remaining space is too small to fit an entry, the writer puts the chunk on a "full list" where it sits until all of the full chunk's entries are obsoleted by txg sync.
Once all entries are obsolete, the full chunk is moved to the "free list" for re-use by a writer.
Each entry header sotres sufficient metadata to
\begin{itemize}
\item attribute entries to a dataset, % via an ID repeated in the ZIL header and each entry's header
\item detect data corruption and missing entries,
\item order entries for replay based on sequence numbers.
\end{itemize}
The physical location of entries in PMEM is irrelevant for replay.
Instead, replay traverses each chunk's entries, filters them for the given dataset and constructs a replay sequence in DRAM using the ordering information stored in the entry headers.

PRB is integrated into the existing ZIL by \lstinline{zilog_pmem_t} which acts as an adaptor between PRB/HDL and the ZIL's "ITX layer".
The ITX layer is shared among all ZIL kinds.
It defines the format of individual log records (i.e. the content of the PRB entry's body) and manages the in-DRAM structures that track which log entries need to be persisted when sync semantics are requested by a syscall.
% for a whole dataset (\lstinline{sync()}), a single file (\lstinline{fsync()}) or an individual syscall (e.g. \lstinline{write()} on an \lstinline{O_SYNC} file descriptors).
However, the actual work of persisting entries is delegated to a ZIL-kind-specific object, i.e., \lstinline{zilog_lwb_t} for ZIL-LWB and \lstinline{zilog_pmem_t} for ZIL-PMEM.
Whereas \lstinline{zilog_lwb_t} implements all of its persistence logic in-line, \lstinline{zilog_pmem_t} is a thin wrapper around the methods of PRB/HDL.

\section{PMEM-aware SPA \& VDEV layer}
\section{ZIL kinds}
\section{PRB/HDL: A generic logical WAL for a zpool}
\section{ZIL-PMEM}
\section{ITXG Bypass For ZVOL}

\chapter{Evaluation}\label{ch:eval}
\section{Correctness}
\subsection{PRB}
\subsection{ZIL-PMEM}
\section{Performance}
\subsection{Write Performance}
\subsubsection{4k Random Sync Writes}
\subsubsection{Application Benchmarks}
\subsection{Replay Performance}

\chapter{Related Work}

\backmatter

\chapter{Appendix}\label{ch:appendix}

\cleardoublepage
\phantomsection
\addcontentsline{toc}{chapter}{Bibliography}
\emergencystretch=1em
\printbibliography

\end{document}
