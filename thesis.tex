\documentclass[12pt,a4paper,twoside,draft]{book}

% use Libertine font
\usepackage{libertine}
\usepackage[T1]{fontenc}

\usepackage[utf8]{inputenc}
\usepackage[inline]{enumitem}
\usepackage{parskip} % disable indentation for new paragraphs, increased margin-bottom instead
\usepackage[ngerman,american]{babel}
\usepackage{csquotes}

\usepackage{kit_style/kitthesiscover}

\usepackage[style=alphabetic,backend=biber]{biblatex}
\addbibresource{bib.bib}
\setcounter{biburlnumpenalty}{100}
\setcounter{biburllcpenalty}{7000}
\setcounter{biburlucpenalty}{8000}

\usepackage{todonotes}
\usepackage{blindtext}

\usepackage{xparse}

\usepackage{xspace}

\usepackage{listings}

% for core allocation pseudo-code mostly...
\usepackage{amsmath}
\usepackage{algorithmicx}
\usepackage{varwidth}
\usepackage{calc} % for \widthof
%\usepackage{algorithm}
\usepackage{algpseudocode}
\usepackage{mathtools}
\DeclarePairedDelimiter\ceil{\lceil}{\rceil}
\DeclarePairedDelimiter\floor{\lfloor}{\rfloor}
\usepackage{amsfonts}
\usepackage{setspace}

% pandas to_latex tables look good this way
\usepackage{booktabs}
\usepackage{multirow}

\usepackage{subcaption}

\usepackage{placeins}

\usepackage{hyperref}

\usepackage{siunitx}


% custom commands
\newcommand{\impl}[0]{$\Rightarrow$}

\widowpenalty100000
\clubpenalty100000
\raggedbottom

\begin{document}
\frontmatter
\unitlength1cm
\selectlanguage{american}

\title{Low-Latency Synchronous I/O For OpenZFS Using Persistent Memory}
\author{Christian Schwarz}
\thesistype{ma}
\primaryreviewer{Prof.\ Dr.\ Frank Bellosa}
\advisor{M.\ Sc.\ Lukas Werling}{}
\thesisbegindate{TODO}
\thesisenddate{TODO}
\maketitle

\thispagestyle{empty}
\vspace*{30\baselineskip}
%\hbox to \textwidth{\hrulefill}
\par
\iflanguage{ngerman}{
\noindent Ich versichere wahrheitsgem"a"s, die Arbeit selbstst"andig verfasst, alle benutzten Hilfsmittel vollst"andig und genau angegeben und alles kenntlich gemacht zu haben, was aus Arbeiten anderer unver"andert oder mit Ab"anderungen entnommen wurde sowie die Satzung des KIT zur Sicherung guter wissenschaftlicher Praxis in der jeweils g"ultigen Fassung beachtet zu haben.\\

\noindent Karlsruhe, den \declaration@date
}{
\noindent I hereby declare that the work presented in this thesis is entirely my own and that I did not use any source or auxiliary means other than these referenced. This thesis was carried out in accordance with the Rules for Safeguarding Good Scientific Practice at Karlsruhe Institute of Technology (KIT).\\

\noindent Karlsruhe, \declaration@date
}

%%%%%%%%%%%%%%%%%%%%%%%%%%%%%%%%%%%%%%%%%%%%%%%%%%%%%%%%%%%%%%%%%%%%%%%%
%% Hinweis:
%%
%% Diese Erklärung wird von der Prüfungsordnung für Diplomarbeiten 
%% verlangt und ist zu unterschreiben. Für Studienarbeiten ist diese
%% Erklärung nicht zwingend notwendig, schadet aber auch nicht.
%%%%%%%%%%%%%%%%%%%%%%%%%%%%%%%%%%%%%%%%%%%%%%%%%%%%%%%%%%%%%%%%%%%%%%%%
\clearpage








\chapter{Abstract}
\blindtext

\mainmatter
\cleardoublepage
\phantomsection
\addcontentsline{toc}{chapter}{Contents}
\tableofcontents

\chapter{Introduction}
The task of a filesystem is to provide non-volatile storage to applications in the form of the \textit{file} abstraction.
Applications modify files through system calls such as \lstinline{write()} which generally does not provide any durability guarantees.
Instead, the system call modifies a buffer in DRAM such as a page in the Linux page cache and returns to userspace.
Synchronization of the dirty in-DRAM data to persistent storage is thus deferred to a --- generally implementation-defined --- point in the future.

However, many applications have more specific durability requirements.
For example, an accounting system that processes a purchase needs to ensure that the updated account balance is persisted to non-volatile storage before clearing the transaction.
Otherwise, a system crash and recovery after clearing could result in the pre-purchase balance being restored, enabling double-spending by the account holder.
These \textbf{synchronous I/O} semantics must be requested through APIs auch as \lstinline{fsync()} which "assure that after a system crash [...] all data up to the time of the \lstinline{fsync()} call is recorded on the disk."~\cite{posix_fsync_opengroup}.

The \textbf{Zettabyte File System (ZFS)} is a combined volume manager and filesystem.
It pools many block devices into a single storage pool (\textit{zpool}) which can hold thousands of sparsely allocated filesystems.
The ZFS on-disk format is a merkle tree that is rooted in the \textit{uberblock} which is ZFS's equivalent of a superblock.
ZFS on-disk state is always consistent and moves forward in so-called \textit{transaction groups} (txg), using copy-on-write to apply updates.
Whenever a new version of the on-disk state needs to be synced to disk, ZFS traverses its logical structure bottom up and builds a new merkle tree.
The updated parts of the tree are stored in newly allocated disk blocks while unmodified parts are re-use the existing block written in a prior txg.
Once all updates have been written out, the new uberblock is written, thereby atomically moving the on-disk format to its new state.
This procedure is called \textit{txg sync} and is triggered periodically (default: every \SI{5}{s}) or if the amount of dirty data exceeds a configurable threshold.

Synchronous I/O semantics cannot be reasonably implemented through txg sync due to the write amplification and CPU overhead inherent to the txg sync procedure.
Instead ZFS maintains the \textbf{ZFS Intent Log (ZIL)} which is a per-filesystem write-ahead log.
Unlike systems such as Linux's \textit{journaling block device 2} (JBD2), the ZIL is a logical log:
the ZIL's records describe the \textit{logical} changes that need to applied in order to achieve the state that was reported committed to userspace.
On disk, the log records are written into a chain of \textit{log-write blocks} (LWBs), each containing many records.
The LWB chain is rooted in the \textit{ZIL header} within the filesystem's representation within the merkle tree.
New LWBs are appended to the chain independently of txg sync.

By default, LWBs are allocated on\todo{from?} the zpool's main storage devices.
Consequently, the lower bound for synchronous I/O latency in ZFS is the time required to write the LWBs that contains the synchronous I/O operation's ZIL records.
For the case where this latency is insufficient, ZFS provides the ability to add a \textit{separate log device} (SLOG) to the zpool.
The SLOG is typically a single (or mirrored) block device that provides lower latency than the main pool's devices.
A typical configuration today is to add a fast NVMe drive to an HDD-based pool.
Adding a fast SLOG accelerates LWB writes because LWBs are preferentially allocated from the SLOG.
Note that SLOGs only need very limited capacity since LWBs are generally obsolete after three txgs.

\textbf{Persistent Memory (PMEM)} is an emerging storage technology that provides low-latency memory-mapped byte-addressable persistent storage.
The Linux kernel can expose PMEM as a pseudo block device whose sectors map directly to the PMEM space.
Thereby existing block devices consumers can benefit from PMEM' low latency without modification.
Block device consumers that wish to bypass block device emulation use the kernel-internal \textit{DAX} API which translates sector numbers to kernel virtual addresses, giving software \textbf{d}irect \textbf{ac}cess to PMEM.

The motivation for this thesis is to accelerate synchronous I/O in ZFS by using PMEM as a ZFS SLOG device.
A single DIMM of the current\todo{product name} Intel Optane PMEM product line can sustain 530k random 4k write IOPS which corresponds to a write latency of \SI{1.88}{us}.
However, when configuring the Linux PMEM block device as a SLOG in OpenZFS 2.0, a single thread only achieves 12k random 4k synchronous write IOPS~($\sim$~\SI{83}{us}), scaling up to 100k IOPS at 16 threads (8 cores, 2xSMT) at doubled latency.
In contrast, the same workload applied directly to the PMEM block device is able to achieve 500k IOPS with two threads~($\sim$~\SI{4}{us} per thread).
And Linux 5.9's ext4 on the PMEM block device achieves 100k random 4k write IOPS ($\sim$~\SI{}) with a single thread and scales approximately linearly to up to 466k IOPS at four threads.
\todo{review numbers in this paragraph}

Our investigation of the latency distribution for the OpenZFS PMEM SLOG configuration shows that the ZIL implementation contributes \SI{48}{us} in the single-threaded case and up to \SI{95}{us} at eight threads.
More specifically, up to \SI{50}{\%}\todo{run this again without ebpf} of total syscall latency is spent waiting for the \textit{ZIO pipeline}, excluding the time spent in the PMEM block device driver.

ZIO is ZFS's abstraction for performing IO to the pool's storage devices.
The pipeline-oriented design has proven extremely flexible and extensible over ZFS's lifetime.
Many of ZFS’s distinguishing features are implemented in ZIO, including physical block allocation, transparent data and metadata checksumming, compression, deduplication and encryption as well as redundancy mechanisms such as raidz.
CPU-intensive processing steps are parallelized to all of the system's CPUs using \textit{taskq}s.

It is our impression that ZIO is biased towards throughput, not latency.
For example, ZIO's main consumer is \textit{txg sync} which issues the updates to the on-disk merkle tree as a large tree of dependent ZIOs.
Other consumers such as periodic data integrity checking (\textit{scrubbing}) are also throughput-oriented.
In contrast, the ZIL is the only component that is latency-oriented.
ZIO's latency overhead with fast block devices such as NVMe drives is a well known problem among ZFS developers.\todo{ref, gh issue, cite private conversation with brian?}

Given
\begin{itemize}[noitemsep]
    \item the abysmal\todo{too harsh?} out-of-the-box performance of the PMEM block device as a SLOG, and
    \item our findings on\todo{right prep?} the ZIL's and ZIO pipeline's significant latency overhead, as well as the fact that
    \item grouping log records into log-write \textit{blocks} is unnecessary with byte-addressable PMEM,
\end{itemize}
we believe that the current ZIL design is fundamentally unfit to reap the performance available with PMEM.
We present \textbf{ZIL-PMEM}, a new implementation of the ZIL that bypasses the ZIO pipeline completely and persists log entries directly to PMEM.
It coexists with the existing ZIL which we refer to as \textit{ZIL-LWB} in the remainder of this document.
ZIL-PMEM achieves 140k random 4k sync write IOPS with a single thread~($\sim$~\SI{7.1}{us}) and scales up to 400k IOPS at 8 threads before it becomes CPU-bound.
This corresponds to a speedup of 6.8x (or 4x, respectively) over ZIL-LWB.\todo{review numbers}
Our implementation is extensively unit-tested and passes the ZFS test suite's SLOG integration tests.

The remainder of this thesis is structured as follows:\todo{related work mid or end}
in chapter~\ref{ch:lwb_analysis}...
%In the following sections, we present
%the user experience of configuring a pool with ZIL-PMEM,
%our strategy for code-sharing between ZIL-PMEM and ZIL-LWB and
%the design of ZIL-PMEM, including
%our strategy for how we integrate PMEM log VDEVs into the vdev / zpool config layer
%ZIL-PMEM’s physical data structure,FEe
%its insertion and garbage collection algorithm,
%its claiming & replay algorithm and
%its integration into the existing OpenZFS software architecture.

\chapter{Background}
Foo \cite{kwon_strata_2017}.
\section{Synchronous I/O}
\section{Persistent Memory}
\section{OpenZFS}

\chapter{Why ZIL-LWB Is Slow On PMEM}\label{ch:lwb_analysis}

\chapter{Design \& Implementation}

\section{Project Scope}
In this section we define the scope of the ZIL-PMEM project.

\newcommand{\csgoal}[1]{\textbf{#1}}

\subsection{Requirements}
The following features are success criteria for the ZIL-PMEM design.
We focus on the functionality and defer our evaluation to chapter~\ref{ch:eval}.

\csgoal{Coexistence}
ZIL-PMEM must coexist with ZIL-LWB due to limited availability of PMEM and limitations of the ZIL-PMEM design.

\csgoal{Same Guarantees}
ZIL-PMEM must maintain the same crash consistency guarantees towards user-space as ZIL-LWB for both ZPL and ZVOL.

\csgoal{Simple Administration \& Pooled Storage}
Pooling of storage resources and simple administration are central to ZFS's design.~\cite{zfspaper}
ZFS should automatically detect that a SLOG device is PMEM and, if so, use ZIL-PMEM for all of the pool’s datasets.
No further administrative action should be required to fully benefit from ZIL-PMEM.

\csgoal{Correctness}
In the absence of PMEM media errors and data corruption, ZIL-PMEM must be able to replay all data that it reported as committed.
The result must be the same as if ZIL-LWB would have been used in lieu.
Specifically:
\begin{itemize}[noitemsep,beginpenalty=100000,midpenalty=100000]
    \item Replay must respect the logical dependencies of log entries.
    \item Logging must be crash-consistent, i.e., the in-PMEM state must always be such that replay is correct.
    \item Replay must be crash-consistent, i.e., if the system crashes or loses power replay must be able to resume. Resumed replay must continue to respect logical dependencies of log entries.
\end{itemize}

\csgoal{Data Integrity}
Data integrity is a core feature of ZFS.~\cite{zfspaper}
ZIL-PMEM must be able to detect corrupted log entries using an error-detecting code.
Detected corruption must be handled correctly in the sense outlined above.
Further, corruption must also be handled gracefully with the following behavior as the baseline.
Assume a sequence of log entries $1 \dots N$ where log entry $1$ does not depend on a log entry and each entry $i > 1$ depends on its predecessor $i-1$.
Data corruption in entry $i \in 1 \dots N$ must not prevent replay of entries $1 \dots i-1$.

\csgoal{Low Latency}
The latency overhead of ZIL-PMEM compared to raw PMEM device latency should be minimal for single threaded workloads.
Multi-threaded workloads are addressed below.

\csgoal{Multi-Core Scalability}
As a pool-wide resource, ZIL-PMEM should scale well to multiple cores.
Barring PMEM throughput limitations, ZIL-PMEM should achieve the following speedups for threads that perform synchronous I/O operations on separate CPU cores:
{
\setlength{\parskip}{0pt}
\begin{description}[topsep=0pt, noitemsep, leftmargin=1cm, labelindent=1cm, widest=1 private dataset per thread]
    \item[1 private dataset per thread] always near-linear speedup
    \item[1 shared dataset] \mbox{}
          \begin{description}[noitemsep, leftmargin=1cm, labelindent=1cm, widest=ZPL file system]
              \item[ZPL file system] no speedup
              \item[ZVOL] up to linear speedup, depending on workload
          \end{description}
\end{description}
}

\csgoal{Maximum Performance on Intel Optane DC Persistent Memory}
Whereas battery-backed NVDIMMs have existed for decades\todo{reference}, Intel Optane DC Persistent Memory is currently the only broadly available flash-based persistent memory product\todo{proof}.
We develop ZIL-PMEM on this platform and want to determine the maximum performance that can be achieved with it.

\csgoal{CPU-Efficient Handling Of PMEM Bandwidth Limits}
PMEM I/O wait time is spent on-CPU --- typically at a memory barrier instruction or because the CPU has exhausted its store or load buffer capacity\todo{review terminology; need proof?}.
\citeauthor{yang_empirical_2020} have shown that a single Optane DIMM's write bandwidth can be exhausted by one CPU core at \SI{2}{GB/s}.
Write bandwidth decreases to \SI{1}{GB/s} at ten or more CPU cores.
In contrast, DRAM shows a near-linear increase in bandwidth to \SI{60}{GB/s} at \SI{15}{threads}.~\cite[fig.4]{yang_empirical_2020}
In practice, these results imply that on-CPU time is wasted as soon as the system writes to PMEM at higher than maximum device bandwidth.
Since ZIL-PMEM shares PMEM among all datasets in the pool, we expect that such overload situtations will happen in practice.
ZIL-PMEM should thus provide a mechanism to shift excessive PMEM I/O wait time off the CPU.

\csgoal{Testability}
ZIL-PMEM must be architected for testability.
The core algorithms presented in this chapter must be covered by unit tests.
Further, ZIL-PMEM should be integrated into the ztest user-space stress test as well as the SLOG tests of the ZFS Test Suite.

\subsection{Out Of Scope For The Thesis}
The following features were omitted to constrain the scope of the thesis.
We believe that our design can accomodate them without major changes.

\csgoal{Support For OpenZFS Native Encryption}
The ZIL-PMEM design presented in this section does not support OpenZFS native encryption.
Intel Optane DC Persistent Memory supports transparent hardware encryption per DIMM at zero overhead\todo{cite spec}.
In contrast, OpenZFS native encryption is per dataset and software-based.
Given these significant differences in data and threat model, ZIL-PMEM cannot rely on Optane hardware encryption.
Instead, ZIL-PMEM would need to invoke OpenZFS native encryption and decryption routines when writing or replaying log entries.

\csgoal{Protection Against Scribbles}
Scribbles are bugs in the system that accidentally overwrite PMEM, e.g., due to incorrect address calculation or out-of-bounds access in the kernel.
PMEM file systems such as NOVA-Fortis have already presented highly-performant mechanisms to protect againt scribbles.~\cite{xu_nova-fortis_2017}
We believe that our design can be trivially extended with similar mechanisms.

\subsection{Limitations}
The following features were deliberately left out of our design.
More experimentation and experience with ZIL-PMEM is necessary to determine which features are useful in practice, how they can be realized and how they interact with the existing requirements.

\csgoal{No NUMA Awareness}
\citeauthor{yang_empirical_2020} recommend to "avoid mixed or multi-threaded accesses to remote NUMA nodes. [...]  For writes, remote Optane’s latency is 2.53x (ntstore) and 1.68x higher compared to local"~\cite{yang_empirical_2020}.
Given the latency contribution of ZIL-PMEM to overall syscall latency (ref~\ref{ch:eval}\todo{precise ref}), variations of this magnitude would be significant.

\csgoal{No Data Redundancy}
ZIL-PMEM provides data integrity protections but does not provide a mechanism for data redundancy.

\csgoal{Only Works With SLOGs}
ZIL-PMEM only works with PMEM SLOGs, not for a zpool with PMEM as main pool vdevs.
Such pools continue to use ZIL-LWB.

\csgoal{No Software Striping}
Our design only supports a single PMEM SLOG device.
Users may wish to use multiple PMEM DIMMs to increase log write bandwidth.
With Intel Optane DC Persistent Memory, multiple PMEM DIMMs can be interleaved in hardware with near-linear speedup.~\cite{yang_empirical_2020}
Whereas software striping would be the natural approach to ZFS, it will be non-trivial to achieve the same speedup as hardware-based interleaving.

\csgoal{No Support For \lstinline{WR_INDIRECT}}
ZIL-LWB logs large write records' data directly to the main pool devices.
The ZIL record then only contains metadata such as \lstinline{mtime} and a block pointer to the location in the main pool.
This technique avoids double-writes which is particularly advantageous if the pool does not have a SLOG, but ZIL-PMEM pools by design always has one.
Further, if the pool has a PMEM SLOG, \lstinline{WR_INDIRECT} log record write latency is likely to be dominated by the main pool's IO latency which likely consists of regular block devices.
If the main pool's IO latency were acceptable, a fast NVMe-based ZIL-LWB SLOG or even no SLOG at all would likely be sufficient for the setup in question.

\csgoal{Space Efficiency}
ZIL-PMEM tends to trade PMEM space for time and simplicity.
Our rationale is that PMEM capacities are significantly higher than DRAM.
For example, the smallest Intel Optane DC Persistent Memory DIMM offered by Intel is 128GiB in size.~\cite{optanepricing_missing}

\section{Overview}
This section summarizes the ZIL-PMEM design at a high level.
The subsequent sections then provide more detail on each abstractions' design and implementation.

We introduce the concept of \textit{ZIL kinds} to ZFS.
The ZIL kind is a pool-scoped variable that determines the pool's strategy for persisting ZIL entries.
A zpool's ZIL kind is determined by the following rule:
if the pool has exactly one SLOG and that SLOG is PMEM, the ZIL kind is ZIL-PMEM. Oterwise, it is ZIL-LWB.
ZIL-LWB is the current ZIL's persistence implementation.
It uses the SPA's metaslab allocator to allocate log-write blocks (LWBs) from the storage pool with a bias towards SLOG devices.
ZIL-PMEM disables metaslab allocation for its PMEM SLOG device and uses the PMEM space directy.

The PMEM space is partitioned into fixed-size segments.
Each segment has a corresponding in-DRAM data structure called \textit{chunk} that tracks the segment's kernel-virtual base address and length.
Chunks are organized in a pool-wide in-DRAM data structure called \textit{PRB}.
PRB implements a high-performance, scalable, crash-consistent, data-corruption-checking and garbage-collected write-ahead log.
It stores log entries in the PMEM chunk's segments as a contiguous append-only sequence.
Full chunks are stashed away and become reusable for logging when the youngest entry's transaction group has synced to the main pool.
After a system crash, a new instance of PRB is instantiated with the same chunks (segments) that the pre-crash PRB instance used.
The new instance scans each chunk for log entries that need to be replayed.
This process is called \textit{claiming}.
The entries' physical location in PMEM is irrelevant for replay.
Instead, each entry stores sufficient metadata to determine whether an entry needs to be replayed, to detect missing entries, and to find a deterministic a replay order.
Once claiming is complete, the new PRB instance is ready for replay and logging by datasets.
It only uses those chunks for logging that do not have a replay claim.

PRB is a pool-wide data structure but the ZIL is written and replayed on a per-dataset basis.
For this purpose, PRB defines a data structure called HDL that holds the per-dataset PRB state, such as the log's GUID, dependency-tracking state and replay position.
All interaction that is scoped to a dataset happens through the HDL, not PRB.
Whereas PRB/HDL implements PMEM persistence, it is the HDL user's responsibility to persist HDL state in the main pool.

The keystone to the ZIL-PMEM architecture is the per-dataset in-DRAM struct \lstinline{zilog_t}.
Before the introduction of ZIL kinds, \lstinline{zilog_t} implemented all ZIL-related functionality.
With ZIL kinds, \lstinline{zilog_t} acts as an abstract base class that encapsulates shared ZIL functionality.
This includes the definitions of the ZIL log record format and the data structures that track which log entries need to be persisted when sync semantics are requested by a syscall.
% % for a whole dataset (\lstinline{sync()}), a single file (\lstinline{fsync()}) or an individual syscall (e.g. \lstinline{write()} on an \lstinline{O_SYNC} file descriptors).
The code that is responsible for persisting log entries resides in ZIL-kind-specific subclasses.
The pool's ZIL kind determines which subclass is instantiated at runtime.
The subclass for ZIL-LWB is \lstinline{zilog_lwb_t} which contains the original LWB code.
The subclass for ZIL-PMEM is \lstinline{zilog_pmem_t}.
It is a thin wrapper around HDL's logging and replay methods.
\lstinline{zilog_pmem_t} persists HDL state in the dataset's ZIL header.
The \lstinline{zil_pmem} module --- which defines \lstinline{zilog_pmem_t} --- is the component that integrates PRB/HDL into ZFS:
it allocates the chunks, constructs the PRB and creates and destroys HDLs in synchrony with their datasets.

\section{PMEM-aware SPA \& VDEV layer}
In this section we describe how we change ZFS to handle PMEM devices.

\section{ZIL kinds}
In this section we describe how we refactor ZFS to allow for different ZIL implementations at runtime.

\section{PRB/HDL: A generic logical WAL for a zpool}
In this section we describe the PRB/HDL data structure in detail.

% PRB log entries consist of an opaque variable-sized body and a plain fixed-size header.
% A log entry is always written to exactly one chunk's PMEM segment.
% The segment is organized as a contiguous append-only sequence of entries that is terminated by an invalid header.
% %Starting with the first entry's header at offset zero, the sequence can be traversed using length information stored in the headers.
% If a chunk's remaining space is too small to fit an entry, the writer puts the chunk on a "full list" where it sits until all of the full chunk's entries are obsoleted by txg sync.
% Once all entries are obsolete, the full chunk is moved to the "free list" for reuse by a writer.
% Each entry header stores sufficient metadata to attribute entries to a dataset, detect data corruption and missing entries, and order entries for replay based on sequence numbers.
% The physical location of entries in PMEM is irrelevant for replay.
% Instead, replay traverses each chunk's entries, filters them for the given dataset and constructs a replay sequence in DRAM using the ordering information stored in the entry headers.

\section{ZIL-PMEM}
In this section we describe how we use PRB/HDL to implement ZIL-PMEM as a new ZIL kind.

\section{ITXG Bypass For ZVOL}
In this section we describe a performance optimization for ZIL-PMEM that allows for parallel log writes to a single ZVOL.

\chapter{Evaluation}\label{ch:eval}
\section{Correctness}
We evaluate the correctness of our work through unit and integration tests.
\subsection{PRB}
\subsection{ZIL-PMEM}
\section{Performance}
We evaluate the performance of ZIL-PMEM with a variety I/O benchmarks.
\subsection{Write Performance}
\subsubsection{4k Random Sync Writes}
\subsubsection{Application Benchmarks}
\subsection{Replay Performance}

\chapter{Related Work}

\backmatter

\chapter{Appendix}\label{ch:appendix}

\cleardoublepage
\phantomsection
\addcontentsline{toc}{chapter}{Bibliography}
\emergencystretch=1em
\printbibliography

\end{document}
